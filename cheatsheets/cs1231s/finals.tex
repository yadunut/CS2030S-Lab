\documentclass[a4paper]{article}
\usepackage[T1]{fontenc}
\usepackage{1231num}
\usepackage{amsmath}
\usepackage{amsthm}
\usepackage{amssymb}

\usepackage{lscape}
\usepackage{pdflscape}

\usepackage[margin=0.5in]{geometry}
\usepackage[skip=0pt]{parskip}
\setlength{\parindent}{0pt}

\theoremstyle{definition}
\newtheorem*{defn}{Defn}

\newtheorem*{propos}{Proposition}
\newtheorem*{corollary}{Corollary}
\newtheorem*{lemma}{Lemma}

\renewcommand{\qedsymbol}{}

\newtheorem{innertheorem}{Theorem}
\newenvironment{theorem}[1]
  {\renewcommand\theinnertheorem{#1}\innertheorem}
  {\endinnertheorem}


\author{Yadunand Prem}

\title{Finals Cheatsheet}

\begin{document}
\section{Tables}

\begin{tabular} {|l|c|c|}
  Commutative  & $p \land q \equiv q \land p$ & $p \lor q \equiv q \lor p$\\
  Associative  & $p \land q \land r \equiv (p \land q) \land r$&\\
  Distributive & $p \land (q \lor r) \equiv (p \land q) \lor (p \land r)$&$p \lor (q \land r) \equiv (p \lor q) \land (p \lor r)$\\
  Identity & $p \land \text{true} \equiv p$ & $p \lor \text{false} \equiv p$\\
  Negation & $p \lor \sim p \equiv \text{true}$ & $p \land \sim p \equiv \text{false}$\\
  Double Negative & $\sim(\sim p) \equiv p$ & \\
  Idempotent & $p \lor p \equiv p$ & $p \land p \equiv p$\\
  Universal bound & $p \lor \text{true} \equiv \text{true}$ & $p \land \text{false} \equiv \text{false}$\\
  de Morgan's & $\sim(p \land q) \equiv \sim p \lor \sim q$ & $\sim(p \lor q) \equiv \sim p \land \sim q$\\
  Absorption & $p \lor (p \land q) \equiv p$ & $p \land (p \lor q) \equiv p$\\
  Implication & $p \Rightarrow q \equiv \sim p \lor q$ & $$\\
  $\sim$(Implication) & $\sim (p \Rightarrow q) \equiv p \land \sim q$ & \\
  \hline & & \\
  Modus Ponens &$p \implies q, p$& $q$ \\
  Modus Tollens &$p \implies q, \sim q$& $\sim p$ \\
  Generalization &$p$& $p \lor q$ \\
  Specialization &$p \land q$& $p$ \\
  Conjunction &$p, q$& $p \land q$ \\
  Elimination &$p \lor q, \sim q$& $p$ \\
  Transitivity &$p \implies q, q \implies r$& $p \implies r$ \\
  Division into cases &$p \land q, p \implies r, q \implies r$& $r$ \\
  Contradiction &$\sim p \implies \text{false}$& $p$ \\
  \hline & & \\
  Commutative  & $A \cup B = B \cup A$ & \\
  Associative  & $(A \cup B) \cup C = A \cup (B \cup C)$ & \\
  Distributive & $A \cup (B \cap C) = (A \cup B) \cap (A \cup C)$ & $A \cap (B \cup C) = (A \cap B) \cup (A \cap C)$\\
  Identity     & $A \cup \emptyset = A$ & $A \cap U = A$\\
  Complement   & $A \cup \bar A = U$ & $A \cap \bar A = \emptyset$\\
  Double Complement & $\bar{\bar A} = A$ & \\
  Idempotent        & $A \cup A = A$ & $A \cap A = A$\\
  Universal Bound   & $A \cup U = U$ & $A \cap \emptyset = \emptyset$ \\
  De Morgan's       & $\overline{A \cup B} = \bar{A} \cap \bar{B}$ & $\overline{A \cap B} = \bar{A} \cup \bar{B}$\\
  Absorption        & $A \cup (A \cap B) = A$ & $A \cap (A \cup B) = A$\\
  Complements of U and $\emptyset$ & $\bar U =\emptyset$ & $\bar \emptyset = U$\\
  Set Difference & $A \setminus B = A \cap \bar B$ &\\
  \hline & & \\
  F1 Commutative & $a + b = b + a$ & $ab = ba$ \\
  F2 Associative & $(a + b)+c = a + (b + c)$ & $(ab)c = a(bc)$ \\
  F3 Distributive & $a(b+c) = ab + ac$ & $(b+c)a = ba + ca$ \\
  F4 Identity & $0 +a = a + 0 = a$ & $1 \cdot a = a \cdot 1 = a $ \\
  F5 Additive inverses & $a + (-a) = (-a) + a = 0$ & \\
  F6 Reciprocals & $a \cdot \frac{1}{a} = \frac{1}{a} \cdot a = 1$ & $a \not = 0$ \\
  \hline & & \\
  T1 Cancellation Add & $a + b = a + c$ & $b = c$ \\
  T2 Possibility of Sub & There is one $x, a + x = b$ & $x = b - a$ \\
  T3 & $b - a = b + (-a)$ & \\
  T4 & $-(-a) = a$ & \\
  T5 & $a(b-c)=ab-ac$ & \\
  T6 & $0 \cdot a = a \cdot 0 = 0$ & \\
  T7 Cancellation Mul & $ab = ac$ & $b = c, a \not = 0$ \\
  T8 Possibility of Div & $a \not = 0, ax = b$ & $x = \frac{b}{a}$ \\
  T9 & $a \not = 0, \frac{b}{a} = b \cdot a^{-1}$ & \\
  T10 & $a \not = 0, (a^{-1})^{-1} = a$ & \\
  T11 Zero Product& $ab = 0 \Rightarrow a = 0 \lor b = 0$ & \\
  T12 Mul with -ve & $(-a)b = a(-b) - -(ab)$ & $-\frac{a}{b} = \frac{-a}{b} = \frac{a}{-b}$\\
  T13 Equiv Frac & $\frac{a}{b} = \frac{ac}{bc}$ & $b \not = 0, c \not = 0$\\
  T14 Add Frac & $\frac{a}{b} + \frac{c}{d} = \frac{ad + bc}{bd}$ & $b \not = 0, d \not = 0$\\
  T15 Mul Frac & $\frac{a}{b} \cdot \frac{c}{d} = \frac{ac}{bd}$ & $b \not = 0, d \not = 0$\\
  T16 Div Frac & $\frac{\frac{a}{b}}{\frac{c}{d}} = \frac{ac}{bd}$ & $b \not = 0, d \not = 0$\\
\end{tabular}

\begin{tabular} {|l|c|c|}
  \hline & & \\
  Ord1 & $\forall a,b \in \mathbb{R}^+$ & $a + b > 0, ab > 0$\\
  Ord2 & $\forall a,b \in \mathbb{R}_{\not = 0}$ & $a$ is positive or negative and not both\\
  Ord3 & 0 is not positive & \\
  $a < b$ & means $b + (-a)$ is positive & \\
  $a \leq b$ & means $a < b$ or $a = b$ & \\
  $a < 0$ & means a is negative& \\
  T17 Trichotomy Law & $a < b \lor b > a \lor a = b$ & \\
  T18 Transitive Law & $a < b$ and $b < c$ & $a < c$\\
  T19 & $a < b$ & $a + c < b + c$ \\
  T20 & $a < b$ and $c > 0$ & $ac < bc$ \\
  T21 & $a \not = 0$ & $a^2 > 0$ \\
  T22 & $1 > 0$ &  \\
  T23 & $a < b$ and $c < 0$ & $ac > bc$ \\
  T24 & $a < b$ & $-a > -b$ \\
  T25 & $ab > 0$ & a and b are both positive or negative \\
  T26 & $a < c$ and $b < d$ & $a+b < c+d$ \\
  T27 & $0 < a < c$ and $<0 < b < d$ & $0 < ab < cd$ \\
\end{tabular}

\section{Math}

\begin{defn}{Even and Odd Integers}\\
  n is even $\Leftrightarrow \exists$ an integer $k$ s.t. $n = 2k$\\
  n is odd $\Leftrightarrow \exists$ an integer $k$ s.t. $n = 2k + 1$
\end{defn}

\begin{defn}{Divisibility}\\
  $n$ and $d$ are integers and $d \not= 0$ \\
  $d | n \Leftrightarrow \exists k \in \mathbb{Z}$ s.t. $n = dk$
\end{defn}

\begin{theorem}{4.2.1} Every Integer is a rational number \end{theorem}

\begin{theorem}{4.2.2} The sum of any two rational numbers is rational \end{theorem}

\begin{theorem}{4.3.1} For all $a, b \in \mathbb{Z}^+$, if $a | b$, then $a \leq b$ \end{theorem}

\begin{theorem}{4.3.2} Only divisors of $1$ are $1$ and $-1$ \end{theorem}

\begin{theorem}{4.3.3} $\forall a, b, c \in \mathbb{Z}$ if $a | b$, $b | c$, $a | c$ \end{theorem}

\begin{theorem}{4.6.1} There is no greatest integer \end{theorem}

\begin{propos}{4.6.4} For all integers $n$, if $n^2$ is even, then $n$ is even. \end{propos}

\begin{defn}{Rational} $r$ is rational $\Leftrightarrow \exists a, b \in \mathbb{Z}$ s.t. $r = \frac{a}{b}$ and $b \not=0$ \end{defn}

\begin{defn}{Fraction in lowest term:} fraction $\frac{a}{b}$ is lowest term if largest $\mathbb{Z}$ that divies both $a$ and $b$ is 1 \end{defn}

\begin{theorem}{4.7.1} $\sqrt{2}$ is irrational \end{theorem}

\section{Logic of Combound Statements}

\begin{theorem}{3.2.1} Negation of universal stmt $\sim(\forall x \in D, P(x)) \equiv \exists x \in D$ s.t. $\sim P(x)$ \end{theorem}

\begin{theorem}{3.2.1} Negation of existential stmt $\sim(\exists x \in D$ s.t. $P(x)) \equiv \forall x \in D, \sim P(x)$ \end{theorem}

\begin{defn}{Contrapositive} of $p \Rightarrow q \equiv \sim q \Rightarrow \sim p$ \end{defn}

\begin{defn}{Converse} of $p \Rightarrow q$ is $q \Rightarrow p$ \end{defn}

\begin{defn}{Inverse} of $p \Rightarrow q$ is $\sim p \Rightarrow \sim q$ \end{defn}

\begin{defn}{Only if:} $p$ only if $q$ means $\sim q \Rightarrow \sim p \equiv p \Rightarrow q$ \end{defn}

\begin{defn}{Biconditional:} $p \Leftrightarrow q \equiv (p \Rightarrow q) \land (q \Rightarrow p)$ \end{defn}

\begin{defn} $r$ is sufficient condition for $s$ means if $r$ then $s$, $r \Rightarrow s$ \end{defn}

\begin{defn} $r$ is necessary condition for $s$ means if $\sim r$ then $\sim s$, $s \Rightarrow r$ \end{defn}

\begin{defn}{Proof by Contradiction}\\ If you can show that the supposition that sttatement $p$ is false leads to a contradiction, then you can conclude that $p$ is true \end{defn}

\section{Methods of Proof}

\begin{tabular} {|c|l|}
  \hline
  Statement & Proof Approach \\
  $\forall x \in D\ P(X)$ & Direct: Pick arbitrary x, prove P is true for that x. \\
                    & Contradiction: Suppose not, i.e. $ \exists x(\sim p)$... Hence supposition $\sim p$ is false (P3) \\
  \hline
  $\exists x \in D\ P(X)$ & Direct: Find x where P is true. \\
                    & Contradiction: Suppose not, i.e. $\forall x (\sim p)$... Hence supposition $\sim p$ is false (P3) \\
  \hline
  $P \Rightarrow Q$ & Direct: Assume P is true, prove Q \\
                    & Contradiction: Assume P is true and Q is false, then derive contradiction \\
                    & Contrapositive: Assume $\sim Q$, then prove $\sim P$ \\
  \hline
  $P \Leftrightarrow Q$ & Prove both $P \Rightarrow Q$ and $Q \Rightarrow P$ \\
  \hline
  $xRy$. Prove R is equivalence  & Prove Reflexive, Symmetric and Transitive \\
  \hline

\end{tabular}

\begin{defn}{Proof by Contraposition}\\
  1. Statement to be proved $\forall x \in D\ (P(x) \Rightarrow Q(x))$\\
  2. Contrapositive Form: $\forall x \in D\ (\sim Q(x) \Rightarrow \sim P(x))$\\
  3. Prove by direct proof\\
  3.1 Suppose x is an element of D s.t. $Q(X)$ is false\\
  3.2 Show that P(x) is false.\\
  4. Therefore, original statement is true
\end{defn}


\section{Set Theory}

\begin{defn}{Set: Unordered collection of objects}\\ Order and duplicates don't matter \end{defn}

\begin{defn}{Membership of Set $\in$: } If $S$ is set, $x \in S$ means $x$ is an element of $S$ \end{defn}

\begin{defn}{Cardinality of Set $|S|$: } The number of elements in $S$ \end{defn}

Common Sets:

$\mathbb{N}$ - Natural Numbers, $\{0, 1, 2\}$

$\mathbb{Z}$ - Integers

$\mathbb{Q}$ - Rational

$\mathbb{R}$ - Real

$\mathbb{C}$ - Complex

$\mathbb{Z}^\pm$ - Positive/Negative Integers

\begin{defn}{Subset}
  $A \subseteq B \Leftrightarrow$ Every element of $A$ is also an element of $B$\\
  $A \subseteq B \Leftrightarrow \forall x(x\in A \Rightarrow x \in B)$
\end{defn}

\begin{defn}{Proper Subset} $A \subsetneq B \Leftrightarrow (A \subseteq B \land A \not = B)$ \end{defn}

\begin{theorem}{6.2.4} An empty set is a subset of every set, i.e. $\emptyset \subseteq A$ for all sets $A$ \end{theorem}

\begin{defn}{Cartesian Product} $A \times B = \{(a, b): a \in A \land b \in B\} $ \end{defn}

\begin{defn}{Set Equality}
  $A = B \Leftrightarrow A \subseteq B \land B \subseteq A$ \\
  $A = B \Leftrightarrow \forall x (x \in A \Leftrightarrow x \in B)$
\end{defn}

\begin{defn}{Union:} $A \cup B = \{x \in U: x \in A \lor x \in B\}$ \end{defn}

\begin{defn}{Intersection:} $A \cap B = \{x \in U: x \in A \land x \in B\}$ \end{defn}

\begin{defn}{Difference:} $B \setminus A = \{x \in U: x \in B \land x \not\in A\}$ \end{defn}

\begin{defn}{Disjoint:} $A \cap B = \emptyset$ \end{defn}

\begin{theorem}{4.4.1} Quotient-Remainder
  $n \in \mathbb{Z}, d \in \mathbb{Z}^+$\\ there exists unique integers q and r such that  $n = dq + r$ and $0 \leq r < d$
\end{theorem}

\begin{defn}{Power Set:} The set of all subsets of $A$, has $2^n$ elements. \end{defn}

\begin{theorem}{6.3.1}
  Suppose $A$ is a finite set with $n$ elements, then $P(A)$ has $2^n$ elements.
  $|P(A)| = 2^{|n|}$
\end{theorem}

\begin{defn}{Cartesian Product of $A_n$}
  $= A_1 \times A_2 \times ... \times A_n = \{(a_1, a_2,...a_n): a_1 \in A_1 \land a_2 \in A_2...$
\end{defn}

\begin{theorem}{6.2.1} Subset Relations
  \begin{numpf*}
    \pfln Inclusion of Intersection: $A \cap B \subseteq A, A \cap B \subseteq B$
    \pfln Inclusion in Union $A \subseteq A \cup B, B \subseteq A \cup B$
    \pfln Transitive Property of Substs: $A \subseteq B \land B \subseteq C \Rightarrow A \subseteq C$
  \end{numpf*}
\end{theorem}

\section{Relations}

\begin{defn} Relation from A to B is a subset of $A \times B$\\
  Given an ordered pair$(x, y) \in A\times B$, $x$ is 
  related to y by $R$ is written $xRy \Leftrightarrow (x, y) \in R$
\end{defn}

\begin{defn} Domain, Co-domain, Range\\
  Let $A$ and $B$ be sets and $R$ be a relation from $A$ to $B$
  \begin{numpf*}
    \pfln Domain of R: is set $\{a \in A: aRb$ for some $b \in B\}$
    \pfln Codomain of R: Set B
    \pfln Range of R: is set $\{b \in B: aRb$ for some $a \in A\}$
  \end{numpf*}
\end{defn}

\begin{defn} Inverse Relation\\
  Let $R$ be a relation from $A$ to $B$, 
  $R^{-1} = \{(y, x) \in B\times A: (x, y) \in R\}$\\
  $\forall x \in A, \forall y \in B ((y, x) \in R^{-1} \Leftrightarrow (x, y) \in R)$
\end{defn}

\begin{defn} 
  Relation on a Set $A$ is a relation from $A$ to $A$.
\end{defn}

\begin{defn} Composition of Relations\\
  A, B and C be sets. $R \subseteq A \times B$ be a relation. $S \subset B \times C$ be relation. Composition of R with S, denoted $S \circ R$ is relation from A to C such that: \\
  $\forall x \in A, \forall z \in C(x S \circ R z \Leftrightarrow (\exists y \in B (xRy \land ySz)))$
\end{defn}

\begin{propos} Composition is Associative
  $A, B, C, D$ be sets. $R \subseteq A \times B$, $S \subseteq B \times C$, $T \subseteq C \times D$\\
  $T \circ ( S \circ R) = T \circ S \circ R$
\end{propos}

\begin{propos} Inverse of Composition
  $A, B, C$ be sets. $R \subseteq A \times B$, $S \subseteq B \times C$\\
  $(S \circ R)^{-1} = R^{-1} \circ S^{-1}$
\end{propos}

\begin{defn} \textbf{Reflexivity, Symmetry, Transitivity}
  \begin{numpf*}
    \pfln Reflexivity: $\forall x \in A (xRx)$
    \pfln Symmetry: $\forall x,y \in A (xRy \Rightarrow yRx)$
    \pfln Transitivity:$\forall x,y,z \in A (xRy \land yRz \Rightarrow xRz)$
  \end{numpf*}
  Refer to proof 6
\end{defn}

\begin{defn} Transitive Closure\\
  Transitive closure of R is relation $R^t$ on A that satiesfies
  \begin{numpf*}
    \pfln $R^t$ is transitive
    \pfln $R \subseteq R^t$
    \pfln If $S$ is any other transitive relation that contains $R$, then $R^t \subseteq S$
  \end{numpf*}
\end{defn}

\begin{defn} Partition\\
  $P$ is partition of set A if
  \begin{numpf*}
    \pfln $P$ is a set of which all elements are non empty subsets of A, $\emptyset \not = S \subseteq A$ for all $S \in P$
    \pfln Every element of $A$ is in exactly on element of P, \\
    $\forall x \in A\ \exists S \in P (x \in S)$ and \\
    $\forall x \in A\ \exists S_1, S_2 \in P(x \in S_1 \land x \in S_2 \Rightarrow S_1 = S_2)$
  \end{numpf*}
  OR $\forall x \in A\ \exists!S \in P(x \in S)$\\
  Elements of a partition are called components
\end{defn}

\begin{defn} Relation Induced by a partition\\
  Given partition $P$ of $A$, the relation $R$ induced by partition: \\
  $\forall x, y \in A, xRy \Rightarrow \exists$ a component of $S$ of $P$ s.t. $x, y \in S$
\end{defn}

\begin{theorem}{8.3.1}[Relation Induced by a Partition]
  Let $A$ be a set with a partition and let R be a relation induced by the partition. Then $R$ is reflexive, symmetric and transitive
\end{theorem}

\begin{defn}[Equivalence Relation]
  $A$ be set and $R$ be relation. $R$ is equivalence relation iff $R$ is reflexive, symmetric and transitive
\end{defn}

\begin{defn} Equivalence Class\\
  Suppose $A$ is set and $\sim$ is equivalence relation on A. For each $A \in A$, equivalence class of $a$, denoted $[a]$ and called class of $a$ is set of all elements $x \in A$ s.t. $a\sim x$\\
  $[a]_{\sim} = \{x \in A: a \sim x \}$
\end{defn}

\begin{theorem}{8.3.4} The partition induced by an Equivalence Relation\\
  If $A$ is a set and $R$ is an equivalence relation on $A$, then distinct equivalence classes of $R$ form a partition of $A$; that is, the union of the equivalence classes is all of $A$, and the intersection of any 2 disctinct classes is empty.
\end{theorem}

\begin{defn} Congruence\\
  Let $a, b \in \mathbb{Z}$ and $n \in \mathbb{Z}^+$. Then $a$ is congruent to $b$ modulo $n$ iff $a - b = nk$, for some $k \in \mathbb{Z}$. In other words, $n | (a - b)$. We write $a \equiv b (\text{mod}\ n)$
\end{defn}

\begin{defn} Set of equivalence classes\\
  Let $A$ be set and $\sim$ be an equivalence relation on $A$. Denote by $A/\sim$, the set of all equivalence classes with respect to $\sim$, i.e.

  $A/\sim = \{[x]_\sim: x \in A\}$
\end{defn}

\begin{theorem}{Equivalence Classes} form a partition
  Let $\sim$ be an equiv. relation on $A$. Then $A/\sim$ is a partition of A.
\end{theorem}

\begin{defn}[Antisymmetry]
  $R$ is antisymmetric iff $\forall x, y \in A(xRy \land yRx \Rightarrow x = y)$ \textit{(DOES NOT IMPLY NOT SYMMETRIC)}
\end{defn}

\begin{defn}[Partial Order Relation]
  $R$ is Partial Order iff R is \textit{reflexive}, \textit{antisymmetric} and \textit{transitive}.
\end{defn}

\begin{defn}{Partially Ordered Set}
  Set A is called poset with respect to partial order relation $R$ on $A$, denoted by $(A, R)$ (Proof 7)
\end{defn}

\begin{defn}{$x \preccurlyeq y$}
  is used as a general partial order relation notation
\end{defn}

\begin{defn}[Hasse Diagram]
  Let $\preccurlyeq$ be a partial order on set $A$. Hasse diagram satisfies the following condition for all distinct $x, y, m \in A$ \\
  If $x \preccurlyeq y$ and no $m \in A$ is s.t. $x \preccurlyeq m \preccurlyeq y$, then x is placed below y with a line joining them, else no line joins $x$ and $y$.
\end{defn}

\begin{defn}[Comparability]
  $a, b \in A$ are comparable iff $a \preccurlyeq b$ or $b \preccurlyeq a$. Otherwise, they are \textbf{noncomparable}
\end{defn}

\begin{defn}[Maximal, Minimal, Largest Smallest]
  Set $A$ be partially ordered w.r.t. a relation $\preccurlyeq$ and $c \in A$
  \begin{numpf}
    \pfln c is maximal element of $A$ iff $\forall x \in A$, either $x \preccurlyeq c$ or $x$ and $c$ are non-comparable. OR $\forall x in A(c \preccurlyeq x \Rightarrow c = x)$
    \pfln c is minimal element of $A$ iff $\forall x \in A$, either $c \preccurlyeq x$ or $x$ and $c$ are non-comparable. OR $\forall x in A(x \preccurlyeq c \Rightarrow c = x)$
    \pfln c is largest element of $A$ iff $\forall x \in A (x \preccurlyeq c)$
    \pfln c is smallest element of $A$ iff $\forall x \in A (c \preccurlyeq x)$
  \end{numpf}
\end{defn}

\begin{propos} A smallest element is minimal\\
  Consider a partial order $\preccurlyeq$ on set $A$. Any smallest element is minimal. 
  \begin{numpf}
    \pfln Let $c$ be smallest elemnt
    \pfln Take any $x \in A$ s.t. $x \preccurlyeq c$
    \pfln By smallestness, we know $c \preccurlyeq x$ too.
    \pfln So $c = x$ by antisymmetry
  \end{numpf}
\end{propos}

\begin{defn}[Total Order Relations] All elements of the set are comparable\\
  R is total order iff $R$ is a partial order and $\forall x, y \in A (xRy \lor yRx)$
\end{defn}

\begin{defn}[Linearization of a partial order]
  Let $\preccurlyeq$ be a partial order on set $A$. A linearization of $\preccurlyeq$ is a total order $\preccurlyeq *$ on $A$ s.t. $\forall x, y \in A (x \preccurlyeq y \Rightarrow x \preccurlyeq *\ y)$
\end{defn}

\begin{defn}[Kahn's Algorithm]
  Input: A finite set $A$ and partial order $\preccurlyeq$ on $A$
  \begin{numpf}
  \pfln Set $A_0 := A$ and $i := 0$
  \pfln Repeat until $A_i = \emptyset$
  \begin{subpf}
    \pfln Find minimal element $c_i$ of $A_i$ wrt $\preccurlyeq$ 
    \pfln Set $A_{i+1} = A_i \setminus {c_i}$
    \pfln Set $i = i+1$
  \end{subpf}
  \end{numpf}

  Output: A linearization $\preccurlyeq *$ of $\preccurlyeq$ defined by setting, for all indicies $i, j$\\ $c_i \preccurlyeq*\ c_j \Leftrightarrow i \leq j$
\end{defn}

\begin{defn}[Well ordered set] Let $\preccurlyeq$ be a total order on set $A$. $A$ is well ordered iff every nonempty subset of A contains a smallest element. OR\\
  $\forall S \in P(A), S \not = \emptyset \Rightarrow (\exists x \in S \forall y \in S (x \preccurlyeq y))$ E.g. $(\mathbb{N}, \leq)$ is well ordered but $(\mathbb{Z}, \leq)$ is not as there is no smallest integer (Theorem 4.6.1)

\end{defn}

\subsection*{Functions}
\begin{defn}[Function] A function f from set $X$ to set $Y$, denoted $f: X \Rightarrow Y$ is a relation satisfying the following\\
  (F1) $\forall x \in X, \exists y \in Y (x,y) \in f$\\
  (F2) $\forall x \in X, \forall y_1, y_2 \in Y(((x, y_1) \in f \land (x, y_2) \in f) \Rightarrow y_1 = y_2)$

  OR

  Let $f$ be a relation on sets $X$ and $Y$, i.e. $f \subseteq X \times Y$. Then $f$ is a function from $X$ to $Y$ denoted $f:X \Rightarrow Y$, iff $\forall x \in X\ \exists! y \in Y (x, y) \in f$
\end{defn}

\begin{defn}[Argument, Image, Preimage, input, output] Let $f:X \Rightarrow Y$ be fn. We write $f(x) = y$ iff $(x, y) \in f$

  $f$ sends/maps x to y is also $x \overset {f}{\Rightarrow} y$ or $f:x \mapsto y$. $x$ is \textbf{argument} of $f$.

  $f(x)$ is read "f of x" or "the \textbf{output} of f for the \textbf{input} x", or "value of $f$ at $x$ or "image of $x$ under $f$"

  If $f(x) = y$, then $x$ is a \textbf{preimage} of y
\end{defn}

\begin{defn}[Setwise image and preimage] Let $f: X \Rightarrow Y$ be a fn from set $X$ to $Y$

  - If $A \subseteq X$, then let $f(A) = \{f(x): x \in A\}$\\
  - If $B \subseteq Y$, then let $f^{-1}(B) = \{x \in X: f(x) \in B\}$\\
  $f(A)$ is the \textbf{setwise image} of $A$ and $f^{-1}(B)$ the \textbf{setwise preimage} of $B$ under $f$. This is \textbf{NOT} the inverse function

  If $f^{-1}(\alpha), \alpha$ is a set, $f^{-1}$ is setwise preimage. else if $x$ member of codomain, $f^{-1}(x)$ is inverse function. $f^{-1}(\alpha)$ need not be function. Use $f^{-1}(\{b\})$ for setwise preimage of single element in codomain
\end{defn}

\begin{defn}[Domain, Co-Domain, Range] Let $f: X \Rightarrow Y$ fn from set $X$ to $Y$.

  X is \textbf{domain} of $f$ and $Y$ the \textbf{co-domain} of $f$.

  \textbf{Range} of $f$ is the (setwise) image of $X$ under f: $\{y \in Y: y = f(x)$ for some $x \in X\}$. Range $\subseteq$ Co-Domain
\end{defn}

\begin{defn}[Sequence] Sequence $a_0,a_1,a_2,...$ can be represented by a function $a$ whos domain is $\mathbb{Z}_{\geq0}$ that satisfies $a(n) = a_n$ for every $n \in \mathbb{Z}_{\geq0}$

  Any function whos domain is $\mathbb{Z}_{\geq m}$ for some $m \in Z$ represents a sequence

  Fibonacci Sequence: $F(0) = 0, F(1) = 1, F(n+2) = F(n+1) + F(n)$
\end{defn}

\begin{defn}[String] Let A be a set. A \textbf{string} or a word over $A$ is an expression in the form of $a_0a_1a_2...a_{l-1}$ where $l \in \mathbb{Z}_{\geq 0}$ and $a_0,a_1,a_2,...,a_{l-1} \in A$. 

  $l$ is called length of string. Empty string $\varepsilon$ is the string of length 0. 

  Let $A*$ denote the set of all strings over $A$
\end{defn}

\begin{defn}[Equality of Sequences] Given two sequences $a_0,a_1,a_2...$ and $b_0,b_1,b_2,...$ defined by fn $a(n) = a_n$ and $b(n) = b_n$ for every $n \in \mathbb{Z}_{\geq0}$, two sequences are equal if and only if $a(n) = b(n)$ for every $n \in \mathbb{Z}_{\geq0}$
\end{defn}

\begin{defn}[Equality of Strings] Given two sequences $s1=a_0a_1a_2...a_{l-1}$ and $s2 = b_0 b_1 b_2, ...,b_{l-1}$ where $l \in \mathbb{Z}_{\geq0}$, we say that $s1 = s2$ if and only if $a_i = b_i$ for all $i \in {0,1,2,...,l-1}$
\end{defn}

\begin{theorem}{7.1.1 Function Equality} 

  Two functions $f: A \Rightarrow B$ and $g: C \Rightarrow D$ are equal if i.e. $f = g$, iff (i) $A = C$ and $B = D$ and (ii) $f(x) = g(x) \forall x \in A$
\end{theorem}

\begin{defn}[Injection] One to one functions: 
  $\forall x_1, x_2 \in X (f(x_1) = f(x_2) \Rightarrow x_1 = x_2)$\\
  or the contrapositive: $x_1 \not = x_2 \Rightarrow f(x_1) \not = f(x_2)$

\end{defn}
\begin{defn}[Surjection] Onto function: 
  $\forall y \in Y \exists x \in X (y = f(x))$\\
  Every element in co-domain has a preimage. So range = co-domain. (Every element in Y has an x)

\end{defn}

\begin{defn}[Bijection] One to one correspondence: 
  $\forall y \in Y\ \exists! x \in X(y = f(x))$
\end{defn}

\begin{defn}[Inverse Functions] Let $f: X \Rightarrow Y$. Then $g: Y \Rightarrow X$ is an \textbf{inverse} of f iff\\
    $\forall x \in X, \forall y \in Y (y = f(x) \Leftrightarrow x = g(y))$
    inverse of $f$ is $f^{-1}$
\end{defn}


\begin{propos}[Uniqueness of Inverse] If $g_1$ and $g_2$ are inverses of $f: X \Rightarrow Y$, then $g_1 = g_2$ (Proof S07L34)
\end{propos}

\begin{theorem}{7.2.3} If $f: X \Rightarrow Y$ is a bijection, then $f^{-1}: Y \Rightarrow X$ is also a bijection. In other words, $f: X \Rightarrow Y$ is bijective iff $f$ has an inverse
\end{theorem}

\begin{defn}[Composition of Functions] Let $f: X \Rightarrow Y$  and $g: Y \Rightarrow Z$ be fns

  $g \circ f: X \Rightarrow Z$ is $(g \circ f)(x) = g(f(x)) \forall x \in X$
\end{defn}

\begin{theorem}{7.3.1} Composition with an Identity Function

  If $f: X \Rightarrow Y$ and $id_x$ is identity fn on $X$ and $id_y$ is identity fn on Y, then
  
  $f \circ id_x = f$ and $id_y \circ f = f$
\end{theorem}

\begin{theorem}{7.3.2} Composition of a Function with its inverse

  If $f: X \Rightarrow Y$ is a bijection with inverse function $f^{-1}: Y \Rightarrow X$, then $f^{-1} \circ f = id_x$ and $f \circ f^{-1} = id_y$
\end{theorem}

\begin{theorem}{Associativity of Function Composition}

  Let $f: A \Rightarrow B, g: B \Rightarrow C, h: C \Rightarrow D$. Then $(h \circ g) \circ f = h \circ (g \circ f)$
\end{theorem}

\begin{defn}[Noncommutativity of Function Composition]
  $(g \circ f) \not = (f \circ g)$
\end{defn}

\begin{theorem}{7.3.3} Composition of Injections

  If $f: X \Rightarrow Y$ and $g: Y \Rightarrow Z$ are both injective, then $g \circ f$ is injective
\end{theorem}
\begin{theorem}{7.3.4} Composition of Surjections

  If $f: X \Rightarrow Y$ and $g: Y \Rightarrow Z$ are both surjective, then $g \circ f$ is surjective
\end{theorem}

\begin{defn}[$\mathbb{Z}/ \sim_n$] The quotient $\mathbb{Z}/\sim_n$ where $\sim_n$ is the congruence-mod-n relation on $\mathbb{Z}$, is denoted $\mathbb{Z}_n$

  E.g. $\mathbb{Z}_3 = \{\{3k:k \in Z\}, \{3k + 1: k \in Z\}, \{3k + 2: k \in Z\}\}$
\end{defn}

\begin{defn}[Addition and Multiplication on $\mathbb{Z}_n$] Whenever $[x], [y] \in \mathbb{Z}_n$ 

  $[x] + [y] = [x + y]$ and $[x] \cdot [y] = [x \cdot y]$
\end{defn}

\subsection*{Function Proofs}

\begin{proof} Prove relation is function: T06Q1 $\forall x, y \in \mathbb{N} (xRy \iff x^2 = y^2)$
  \begin{numpf*}
    \pfln $\forall x \in \mathbb{N}, \exists y = x \in \mathbb{N}$ such that $(x,y) \in R$ (F1)
    \pfln F2
    \begin{subpf}
      \pfln $\forall x \in \mathbb{N}$, let $y_1, y_2 \in \mathbb{N}$
      \pfln Suppose $(x,y_1) \in R \land (x, y_2) \in R$
      \pfln Then $y_1^2 = x^2$ and $y_2^2 = x^2$ (by defn of R)
      \pfln Then $y_1^2 = y_2^2$
      \pfln Hence $y_1 = y_2$ (as $y_1,y_2 \in \mathbb{N} > 0$)
    \end{subpf}
  \end{numpf*}
\end{proof}

\begin{proof} Proof of Injection: T06Q2 $f(x) = x+3$
  \begin{numpf*}
    \pfln Let $x_1, x_2 \in \mathbb{R}$ such that $f(x_1) = f(x_2)$
    \pfln Then $x_1 + 3 = x_2 + 3$
    \pfln Then $x_1 = x_2$, therefore $f$ is injective
  \end{numpf*}
\end{proof}

\begin{proof} Proof of Surjection: T06Q2 $f(x) = x+3$
  \begin{numpf*}
    \pfln Take any $y \in \mathbb{R}$
    \pfln Let $x = y - 3$
    \pfln Then $f(x) = f(y-3) = (y-3)+3 = y$, Therefore, $f$ is surjective
  \end{numpf*}
\end{proof}
\begin{proof} Proof of Bijection via Inverse T06Q5: $f(x) = 12x+31$
  \begin{numpf*}
    \pfln $\forall x,y \in \mathbb{Q}, y = 12x + 31 \iff x = (y-31)/12$
    \pfln define $g: \mathbb{Q} \to \mathbb{Q}$ by setting, $\forall y \in \mathbb{Q}, g(y) = (y-31)/12$
    \pfln Then whenever $x,y \in \mathbb{Q}, y=f(x) \iff x = g(y)$
    \pfln Thus $g$ is the inverse of $f$, hence $f$ is bijective (by Theorem 7.2.3)
  \end{numpf*}
\end{proof}


\subsection*{Mathematical Induction}

\begin{defn}[Sequence] Ordered Set with members called \textbf{terms}. May have infinite terms. In the form: $a_m, a_{m+1}, a_{m+2}, ...$
\end{defn}

\begin{defn}[Summation] 
  if $m$ and $n$ are integers and $m \leq n$, $\sum_{k=m}^{n}a_k$ is the sum of all terms $a_m, a_{m+1},...,a_n$

  $k$ is the \textbf{index} of summation, $m$ is the \textbf{lower limit} and n the \textbf{upper limit}

  $\sum^{m}_{k=m}a_k = a_m$ and $\sum^{n}_{k=m}a_k = (\sum^{n-1}_{k=m}a_k) + a_n$
\end{defn}

\begin{defn}[Product] 
  if $m$ and $n$ are integers and $m \leq n$, $\prod_{k=m}^{n}a_k$ is the product of all terms $a_m, a_{m+1},...,a_n$

  $\prod^{m}_{k=m}a_k = a_m$ and $\prod^{n}_{k=m}a_k = (\prod^{n-1}_{k=m}a_k) \cdot a_n$
\end{defn}

\begin{theorem}{5.1.1} Properties of Summations and Products
  \begin{enumerate}
    \item $\sum^{n}_{k=m}a_k + \sum^n_{k=m}b_k = \sum^n_{k=m}(a_k + b_k)$
    \item $c \cdot \sum^{n}_{k=m}a_k = \sum^n_{k=m}(c \cdot b_k)$
    \item $(\prod^{n}_{k=m}a_k) \cdot (\prod^n_{k=m}b_k) = \prod^n_{k=m}(a_k \cdot b_k)$
  \end{enumerate}
\end{theorem}

\begin{defn} Arithmetic Sequence
  $a_0, a_1,a_2$ is arithmetic if there is a constant d s.t. $a_k = a_{k-1}+d$ for all integers $k \geq 1$\\ 
  It follows that $a_n = a_0 + dn$ for all integers $n \geq 0$. $d$ is the common difference. $\sum^{n-1}_{k=0}a_k = \frac{n}{2}(2a_0 + (n-1)d)$
\end{defn}
\begin{defn} Geometric Sequence
  $a_0, a_1,a_2$ is arithmetic if there is a constant r s.t. $a_k = ra_{k-1}$ for all integers $k \geq 1$\\ 
  It follows that $a_n = a_0 r^n$ for all integers $n \geq 0$. $r$ is the common ratio. $\sum^{n-1}_{k=0}a_k = a_0(\frac{1-r^n}{1-r})$
\end{defn}

\begin{defn} Principle of Mathematical Induction

  To prove that "For all integers $n \geq a, P(n)$ is true"
  \begin{itemize}
    \item \textbf{Basis Step:} Show that $P(a)$ is true.
    \item \textbf{Inductive Step: } Show that for all integers $k \geq a, P(k) \implies p(k+1)$. To perform this, suppose that P(k) is true, where k is a particular but arbitrarily chosen integer $k \geq a$
    \item Therefore $P(n)$ is true for all $n \in \mathbb{Z}^+$
  \end{itemize}
\end{defn}

\begin{theorem}{5.2.2} Sum of first n integers: for all integers $n \geq 1, 1 + 2 + 3 + ... + n = \frac{n(n+1)}{2}$ \end{theorem}

\begin{theorem}{5.2.3} Sum of a geometric sequence: for any real number $r \not = 1$, and any integers $n \geq 0, \sum^{n}_{i=0} r^i = \frac{r^{n+1}-1}{r-1}$ \end{theorem}

\begin{propos}{5.3.1} For all integers $n \geq 0, 2^{2n}-1$ is divisible by 3 \end{propos}

\begin{defn}[Strong induction (2PI)] If
  \begin{itemize}
    \item $P(a)$ holds
    \item For every $k \geq a$, $(P(a) \land P(a+1) \land ... \land P(k)) \Rightarrow P(k+1)$
  \end{itemize}
  Then $P(n)$ holds for all $n \geq a$
\end{defn}

\begin{defn}[Strong Induction Variant (2PI)] If
  \begin{itemize}
    \item $P(a), P(a+1),...,P(b)$ holds
    \item For every $k \geq a, P(k) \Rightarrow P(k+b-a+1)$
  \end{itemize}
  Then $P(n)$ holds for all $n \geq a$
\end{defn}

\begin{defn}[Well-Ordering Principle]
  Every nonempty subset of $\mathbb{Z}_{\geq 0}$ has a smallest element
\end{defn}

\begin{defn}[Recurrance Relation] for a sequence $a_0, a_1, a_2,...$ is a formula that relates each term $a_k$ to certain of its predecessors $a_{k-1},a_{k-2},...,a_{k-i}$, where $i$ is an integer with $k-i \geq 0$\\ If $i$ is a fixed integer, the \textbf{initial conditions} for such a recurrant relation specify the values of $a_0,a_1,a_2,...,a_{i-1}$\\
  If $i$ depends on $k$, the initial conditions specify the values of $a_0,a_1,a_2,...,a_{m}$, where $m$ is an integer with $m \geq 0$\\
  E.g. Fibonacci: $F_0 = 0; F_1 = 1; F_n = F_{n-1} +F_{n-2}$, for $n > 1$
\end{defn}

\begin{defn}[Recusively Defined Sets]
  Let $S$ be a finite set with at least 1 element. A \textbf{string over} S is a finite sequence of elements from S. The elements of S are called \textbf{characters} of the string, and the length of a string is the number of characters it contains. The \textbf{null string over} S is defined to be the string with no characters (Length 0, $\varepsilon$).\\
  E.g.
  \begin{enumerate}
    \item Base: $()$ is in $P$
    \item Recusion: 
      \begin{enumerate}
        \item If $E$ is in $P$, so is (E).
        \item If $E$ and $F$ are in $P$, so is $EF$
      \end{enumerate}
    \item Restriction: No configuration of parentheses are in $P$ other than those derived from 1 and 2 above.
  \end{enumerate}
\end{defn}

\begin{defn}[Recursive definition of a set $S$]\ 
  \begin{itemize}
    \item (base clause) - Specify that certain elements, called \textbf{founders} are in $S$: if $c$ is a founder, then $c \in S$
    \item (recursion clause) - Specify certain functions, called \textbf{constructors} under which set $S$ is closed: if $f$ is a constructor and $x \in S$, then $f(x) \in S$
    \item (minimality clause) - Membership for $S$ can always be demonstrated by (infinitely many) successive applications of the clauses above
  \end{itemize}
\end{defn}

\subsection*{Mathematical Induction Proofs}

\begin{proof}1PI Example: Given any set $A, |P(A)| = 2^n$, where P(A) is power set of A and |A| = n.
  \begin{numpf}
    \pfln For each $n \in \mathbb{N}$, let $P(n) \equiv (|P(A)| = 2^n$, where A is any n-element set
    \pfln Basis Step: P(0) is true because $|P(\emptyset)| = |\{\emptyset\}| = 1 = 2^0$ as $P(\emptyset) = \{\emptyset\}$ and $|\emptyset| = 0$
    \pfln Induction Step:
    \begin{subpf}
      \pfln Let $k \in \mathbb{N}$ such that P(k) is true, i.e. $|P(X)| = 2^k$, where X is any k-element set
      \pfln Let A be a $k+1$ element set.
      \pfln Since $k \geq 0$, there is at least one element in A. Pick $z \in A$.
      \pfln The subsets of A can be split to 2 groups: those that contain z and those that don't
      \pfln Subsets that don't contain z are the same as the subsets of $A \setminus \{z\}$, which has a cardinality of k, and hence $|P(A\setminus \{z\})| = 2^k$ (by induction hypothesis)
      \pfln Those subsets that contain z can be matched up one for one with those that do not contain z by unioninzing $\{z\}$ to the latter
      \pfln Hence there is equal no of subsets that contain z and subsets that don't
      \pfln Hence $|P(A)| = 2^k + 2^k = 2^{k+1}$
      \pfln Thus, P(k+1) is true
    \end{subpf}
    \pfln Therefore $\forall n \in \mathbb{N}, P(n)$ is true by MI
  \end{numpf}

\end{proof}

\begin{proof}2PI example: Any integer greater than 1 is divisible by a prime number
  \begin{numpf*}
    \pfln Let $P(n) \equiv (n$ is divisible by a prime), for $n > 1$
    \pfln Basis Step: $P(2)$ is true since 2 is divisible by 2
    \pfln Inductive step To show that for all integers $k \geq 2$, if $P(i)$ is true, for all integers $i$ from 2 to $k$, then $P(k+1)$ is also true.
    \begin{subpf}
    \pfln Case 1 (k+1) is prime: in this case, K+1 is divisible by prime number, itself
    \pfln Case 2 (k+1) is not prime: In this case, $k+1 = ab$, $a$ and $b$ are integers with $1 < a < k+1$ and $1 < b < k+1$
    \begin{subpf}
      \pfln Thus, in particular, $2 \leq a \leq k$ and so by inductive hypothesis, a is divisble by prime number $p$
      \pfln In addition, because $k+1 = ab$, so $k+1$ is divisible by $a$
      \pfln By transitivity of divisibility, $k+1$ is divisible by prime $p$
    \end{subpf}
    \end{subpf}
    \pfln Therefore any integer greater than 1 is divisible by prime
  \end{numpf*}
\end{proof}

\begin{proof} 2PI for Sums: Prove that for any positive int n, if $a_1, a_2, ..., a_n$ and $b_1, b_2, ..., b_n$ are $\mathbb{R}$, then $\sum^n_{i=1}(a_i+b_i) = \sum^n_{i=1}(a_i) + \sum^n_{i=1}(b_i)$
  \begin{numpf*}
    \pfln Let P(n) = $(\sum^n_{i=1}(a_i+b_i) = \sum^n_{i=1}(a_i) + \sum^n_{i=1}(b_i))$, for $n \geq 1$
    \pfln Basis Step: P(1) is true since $\sum^{1}_{i=1}(a_i+b_i) = a_i + b_i = \sum^{1}_{i=i}a_i + \sum^{1}_{i=i}b_i$
    \pfln Inductive Hypothesis: for some $k \geq 1$, $\sum^k_{i=1}(a_i+b_i) = \sum^k_{i=1}(a_i) + \sum^k_{i=1}(b_i)$
    \pfln Inductive Step = $\sum^{k+1}_{i=1}(a_i+b_i) = \sum^k_{i=1}(a_i+b_i) + (a_{k+1} + b_{k+1})$ (By defn of $\sum$)\\
    $ = \sum^{k}_{i=i}a_i + \sum^{k}_{i=i}b_i + (a_{k+1} + b_{k+1})$ (by inductive hypothesis) \\
    $ = \sum^{k}_{i=i}a_i + a_{k+1} + \sum^{k}_{i=i}b_i + b_{k+1}$ (by assoc and commutative law of algebra)\\
    $ = \sum^{k+1}_{i=i}a_i  \sum^{k+1}_{i=i}b_i$ (By defn of $\sum$)
    \pfln Therefore, P(k+1) is true, therefore P(n) is true for any positive integer n
  \end{numpf*}
\end{proof}

\subsection*{Cardinality}

\begin{defn}[Pigeonhole Principle]
  Let $A$ and $B$ be finite sets. If there is an injection $f: A \Rightarrow B$, then $|A| \leq |B|$\\
  Contrapositive: Let $m,n \in \mathbb{Z}^+$ with $m > n$. If $m$ pigeons are put into $n$ pigeonholes, there must be (at least) one pigeonhole with (at least) two pigeons.
\end{defn}
\begin{defn}[Dual Pigeonhole Principle]
  Let $A$ and $B$ be finite sets. If there is an surjection $f: A \Rightarrow B$, then $|A| \geq |B|$\\
  Contrapositive: Let $m,n \in \mathbb{Z}^+$ with $m < n$. If $m$ pigeons are put into $n$ pigeonholes, there must be (at least) one pigeonhole with no pigeons.
\end{defn}

\begin{defn}[Finite set and Infinite Set]
  Let $\mathbb{Z}_n = \{1, 2, 3, ..., n\}$, the set of positive integers from 1 to $n$.\\
  A set $S$ is said to be \textbf{finite} iff $S$ is empty, or there exists a bijection from $S$ to $\mathbb{Z}_n$ for some $n \in \mathbb{Z}^+$\\
  A set $S$ is said to be \textbf{infinite} if it is not finite
\end{defn}

\begin{defn}[Cardinality]
  Cardinality of a finite set $S$, denoted $|S|$, is\\
  (i)  0 if $S = \emptyset$, or\\
  (ii) n if $f: S \Rightarrow Z_n$ is a bijection
\end{defn}

\begin{theorem}{Equality of Cardinality of Finite Sets} Let A and B be any finite sets.\\
  $|A| = |B|$ iff there is a bijection $f: A \Rightarrow B$
\end{theorem}

\begin{defn}[Same Cardinality (Cantor)]
  Given any 2 sets $A$ and $B$. A is said to have the same cardinality as $B$, $|A| = |B|$, iff there is a bijection $f: A \Rightarrow B$
\end{defn}

\begin{theorem}{7.4.1 Properties of Cardinality} Cardinality is an equivalence relation
  \begin{itemize}
    \item \textbf{Reflexive}: $|A| = |A|$
    \item \textbf{Symmetric}: $|A| = |B| \Rightarrow |B| = |A|$
    \item \textbf{Transitive}: $(|A| = |B|) \land (|B| = |C|) \Rightarrow |A| = |C|$
  \end{itemize}
\end{theorem}

\begin{defn}[Cardinal Numbers] Define $\aleph_0 = |\mathbb{Z}^+|$ \end{defn}
\begin{defn}[Coutably Infinite] Set S is said to be countably infinite iff $|S|  = \aleph_0$ \end{defn}
\begin{defn}[Coutably Infinite] Set S is said to be countable iff it is finite or countably infinite \end{defn}

\begin{defn}[$\mathbb{Z}$ is countable]
  $f(n) = \begin{cases}
    n/2, & \text{if n is an even positive integer}\\
    -(n-1)/2, & \text{if n is an odd positive integer}\\
  \end{cases}$
\end{defn}

\begin{defn}[$\mathbb{Q}^+$ is countable] \end{defn}
\begin{defn}[$\mathbb{Z}^+ \times \mathbb{Z}^+$ is countable] \end{defn}

\begin{theorem}[Cartesian Product] If sets $A$ and $B$ are both countably infinite, then so is $A \times B$.\end{theorem}
\begin{corollary}[General Cartesian Product] Given $n \geq 2$ countably infinite sets $A_1, A_2, ..., A_n$, cartesian product $A_1 \times A_2 \times ... \times A_n$ is also countably infinite \end{corollary}

\begin{theorem}[Unions] Union of countably many countable sets is also countable. \end{theorem}

\begin{propos}[9.1] Infinite set B is countable if and only if there is a sequence $b_0, b_1, ... \in B$ in which every element of $B$ appears exactly once \end{propos}
\begin{lemma}[9.2] Infinite set B is countable if and only if there is a sequence $b_0, b_1, ...$ in which every element of $B$ appears \end{lemma}

\begin{theorem}{7.4.2}[Cantor] Set of real numbers between 0 and 1, $(0,1) = \{x\in \mathbb{R} | 0 < x < 1\}$ is uncountable \end{theorem}
\begin{theorem}{7.4.3} Any subset of any countable set is countable \end{theorem}
\begin{corollary}[7.4.4 (Contrapositive of 7.4.3)] Any set with an uncountable subset is uncountable\end{corollary}
\begin{propos}[9.3] Every infinite set has a countably infinite subset \end{propos}
\begin{lemma}[9.4 Union of countably infinite sets] Let A and B be countably infinite sets. Then $A \cup B$ is countable \end{lemma}

\subsection*{Counting and Probability}
\begin{defn}[Sample Space] is set of all possible outcomes of random process or experiment \end{defn}
\begin{defn}[Event] is subset of sample space \end{defn}

\begin{defn}[Probability of Event E in Sample Space S] $P(E) = \frac{|E|}{|S|}$, where |E| is number of outcomes in E and |S| is total number of outcomes \end{defn}

\begin{theorem}{9.1.1}[Number of Elements in a List]
  If $m$ and $n$ are integers and $m \leq n$, then there are $n-m+1$ integers from $m$ to $n$ inclusive.
\end{theorem}

\begin{theorem}{9.2.1}[Multiplication/Product Rule]
  If operation consists of k steps and 1st step performed in $n_1$ ways\\
  2nd step in $n_2$ ways, $k^{th}$ step can be done in $n_k$ ways

  Entire Operation in $n_1 \times n_2 \times ... \times n_k$ ways. 

  Should only be used for independent events
\end{theorem}

\begin{theorem}{9.2.2}[Permutations] Number of permutations of a set with $n (n \geq 1)$ elements is $n!$ (Ordered selection) \end{theorem}

\begin{defn}[R-Permutation] of a set of n elements is an ordered selection of $r$ elements taken from the set. Number of r-permutations of a set of n elements is $P(n,r)$ \end{defn}

\begin{theorem}{9.2.3}[r-permutation from a set of n elements] If n and r are integers and $1 \geq r \geq n$, then number of r-permutations fo a set n is given by $P(n,r) = n(n-1)(n-2)...(n-r+1)= \frac{n!}{(n-r)!}$ \end{theorem}

\begin{theorem}{9.3.1}[Addition/Sum Rule] Suppose a finite set $A$ equals the union of k distinct mutually disjoint subsets $A_1, A_2, ..., A_k$. Then $|A| = |A_1| + |A_2| + ... + |A_k|$ \end{theorem}

\begin{theorem}{9.3.2}[The Difference Rule] if A is a finite set and $B \subseteq A$, then $|A \setminus B| = |A| - |B|$ \end{theorem}

\begin{theorem}[Probability of complement of event] If S is a finite space and A is an event in S, then $P(\bar A) = 1 - P(A)$ \end{theorem}

\begin{theorem}{9.3.3}[Inclusion/Exclusion Rule for 2/3 sets]
  If A, B and C are finite sets, then $|A \cup B| = |A| + |B| - |A \cap B|$ and $|A \cup B \cup C| = |A| + |B| + |C| - |A \cap B| - |A \cap C|  - |B \cap C| + |A \cap B \cap C|$
\end{theorem}

\begin{theorem}[Pigeonhole Principle (PHP)] Function from one finite set to a smaller finite set cannot be one-to-one. There must at least be 2 other elements in the domain that have same image in codomain \end{theorem}

\begin{theorem}[Generalised PHP] For any function $f$ from finite set $X$ with $n$ elements to a finite set $Y$ with $m$ elements and for any positive integer $k$, if $k < n/m$, then there is some $y \in Y$ s.t. $y$ is the image of at least $k+1$ distinct elements of $X$. \end{theorem}

\begin{theorem}[Generalised PHP (Contrapositive)] For any function $f$ from finite set $X$ with $n$ elements to a finite set $Y$ with $m$ elements and for any positive integer $k$, if for each $y \in Y, f^{-1}(\{y\})$ has at most $k$ elements, then $X$ has at most $km$ element; in other words $n \leq km$ \end{theorem}

\begin{defn}[R-combination] Let $n$ and $r$ be non-negative intgers with $r \leq n$. An r-combination of a set of $n$ elements is a subset of $r$ of the $n$ elements. (Unordered selection)

  $n \choose r$, read "n choose r" denotes no of subsets of size $r$ that can be chosen from a set of $n$ elements.
\end{defn}

\begin{defn}[Relationship between Permutation and Combination] 
  To get permutations of $\{0,1,2,3\}$, 
  \begin{enumerate}
    \item Write the 2-combinations of $\{0, 1, 2, 3\}$ --> $(0,1), (0,2), (0,3),(1,2), (1,3),(2,3)$
    \item Order the 2 combination to obtain 2 permutations: $(0,1)$ and $(1,0)$, etc
  \end{enumerate}

  Therefore, $P(n,r) = {n \choose r} \cdot r! = \frac{n!}{(n-r)!}$
\end{defn}

\begin{theorem}{9.5.1}[Formula for $n \choose r$]
  $ = \frac{P(n, r)}{r!} = \frac{n!}{r!(n-r)!}$
\end{theorem}

\begin{theorem}{9.5.2}[Permutations of sets of indistinguishable objects] Suppose collection consists of $n$ objects of which

  $n_1, n_2, ..., n_k$ are of types \{1,2,...,k\} and indistinguishable from each other
  
  and suppose that $n_1 + n_2 + ... + n_k = n$. \\
  Then number of distinguisiable permutations = ${n \choose n_1}{n-n_1 \choose n_2}{n-n_1-n_2 \choose n_3}...{n-n_1-n_2-...-n_k-1 \choose n_k} = \frac{n!}{n_1!n_2!...n_k!}$
\end{theorem}

\begin{defn}[Example of 9.5.2] Order letters in MISSISSIPPI, how many orders are there?
    
  Subset of 4 positions for S = $11 \choose 4$, 4 positions for I = $7 \choose 4$, 2 positions for P = $3 \choose 2$, 1 positions for M = $1 \choose 1$,  ${11 \choose 4}{7 \choose 4}{3 \choose 2}{1 \choose 1} = \frac{11!}{4!4!2!1!}$
\end{defn}

\begin{defn}[Multiset] An r-combination with repitition allowed, or multiset of size $r$, chosen from a set of $X$ of $n$ elements is an unordered selection of elements taken from $X$ with repetition allowed. If $X = \{x_1, x_2, ...,x_n\}$, we write an r-combination with repetition allowed as $[x_{i_1}, x_{i_2},..., x_{i_r}]$ where each $x_{i_j}$ is in $X$ and some of the $x_{i_j}$ may equal each other. \end{defn}

\begin{theorem}{9.6.1}[Number of r-combinations with Repetition Allowed] (multisets of size r) that can be selected from a set of $n$ elements is ${r+n-1} \choose r$ = number of ways r objects can be selected from n categories of objects with repetitions allowed
\end{theorem}

\begin{defn} Which formula to use?
  \begin{tabular} { |l|c|c| }
    \hline
    & Order Matters & Order Does Not Matter \\
    \hline
    Repetition & $n^k$ & $k+n-1 \choose k$ \\
    No Repetition & $P(n, k)$ & $n \choose k$ \\
    \hline
  \end{tabular}
\end{defn}

\begin{theorem}{9.7.1}[Pascals Formula] Let $n$ and $r$ be positive integers, $r \leq n$. Then ${n+1 \choose r} = {n \choose r-1} + {n \choose r}$
\end{theorem}

\begin{defn} Combinations
  \begin{numpf}
    \pfln For $0 \leq k \leq n, {n \choose k} = {n \choose n-k}$
    \pfln For $0 \leq k \leq n, k{n \choose k} = n{n-1 \choose k-1}$
  \end{numpf}
\end{defn}

\begin{theorem}{9.7.2} Binomial Theorem
  Given any real numbers $a$ and $b$ and any non-negative integer $n$,\\ 
  $(a+b)^n = \sum\limits^{n}_{k=0}{n \choose k} a^{n-k}b^k$
\end{theorem}

\begin{theorem}[Probability Axioms] P is a probability function from the set of all events in S.
  \begin{enumerate}
    \item $0 \geq P(A) \geq 1$
    \item $P(\emptyset) = 0$ and $P(S) = 1$
    \item If $A$ and $B$ are disjoint events, $(a \cap B = \emptyset)$, then $P(A \cup B) = P(A) + P(B)$
  \end{enumerate}
\end{theorem}

\begin{defn}[Probability of General Union of 2 events] If A and B are events in S, then $P(A \cup B) = P(A) + P(B) - P(A \cap B)$
\end{defn}

\begin{defn}[Expected Value]  $ = \sum^{n}_{k=1}a_kp_k = a_1p_1 + a_2p_2 + ... + a_np_n$, where a is outcome and p is probability of outcome \end{defn}

\begin{defn}[Linearity of Expectation] Expected Value of sum of random variables x and y = $E[X + Y] = E[X] + E[Y]$, 
\end{defn}

\begin{defn}[Conditional Probability] of B given A, $P(B|A) = \frac{P(A \cap B)}{P(A)}$ \end{defn}

\begin{theorem}{9.9.1}[Bayes' Theorem]
  Sample space S is union of mutually disjoint events $B_1,B_2,...,B_n$ and 
  Suppose A is an event in S, and suppose $P(A) \not = 0$ and $P(B_i) \not = 0$. 

  $P(B_k|A) = \frac{P(A|B_k) \cdot P(B_k)}{P(A|B_1)\cdot P(B_1)+ P(A|B_2)\cdot P(B_2) + ... + P(A|B_n)\cdot P(B_n)} = \frac{P(A|B_k)\cdot P(B_k)}{P(A)}$
\end{theorem}

\begin{defn}[Independent Event] If A and B are events in S, then A and B are independent, if and only if $P(A \cap B) = P(A) \cdot P(B)$
\end{defn}

\begin{defn}[Pairwise Independent and Mutually Independent] A, B and C are events in S. A, B, C are pairwise independent iff they satisfy conditions 1-3. Mutually independent iff all 4 conditions satisfied

  \begin{enumerate}
    \item $P(A \cap B) = P(A) \cdot P(B)$
    \item $P(A \cap C) = P(A) \cdot P(C)$
    \item $P(B \cap C) = P(B) \cdot P(C)$
    \item $P(A \cap B \cap C) = P(A) \cdot P(B) \cdot P(C)$
  \end{enumerate}

\end{defn}

\subsection*{Graphs}

\begin{defn}[Undirected Graph] 2 finite sets: Nonempty set of vertices V, set of edges, where each edge is associated with 1 or 2 vertices. \\
Adjacent Vertice - 2 vertices connected by edge\\
Adjacent Edges - 2 edges incident on same endpoint
\end{defn}

\begin{defn}[Directed Graph] Same as undirected but has set of Directed Edges E, where each edge is an ordered pair of vertices \end{defn}

\begin{defn}[Simple Graph] is undirected graph without any loops or parallel edges \end{defn}
\begin{defn}[Complete Graph] on n vertices, $n > 0, K_n$ is simple graph with n vertices and exactly 1 edge connecting each pair of distinct vertices (All of the nodes are directly connected) \end{defn}
\begin{defn}[Bipartite Graph] is simple graph whose vertices can be divided to 2 disjoint sets U and V such that every edge connects U to one in V \end{defn}

\begin{defn}[Complete Bipartite Graph] is bipartite graph on 2 disjoint sets U and V such that every vertex in U connects to every in Vertex in V. If |U| = m, |V| = n, complete bipartite graph is $K_{m,n}$ \end{defn}

\begin{defn}[Subgraph of a Graph] H is subgraph of G iff every vertex in H is in G, every edge in H is in G, every edge in H has same endpoints as G \end{defn}

\begin{defn}[Degree of Vertex] Degree of v, $deg(v)$ = number of edges incident on v, with loops counted twice.\\ 
  Total degree of G, $deg(G)$ = sum of all degrees of all vertices in G
\end{defn}

\begin{theorem}{10.1.1}[Handshake Theorem] If G is any graph, $deg(G) = deg(v_1) + deg(v_2) + ... + deg(v_n) = 2 \times |E|$, where E is the set of edges in G. \end{theorem}
\begin{corollary}{10.1.2} Total Degree of a graph is even \end{corollary}
\begin{propos}{10.1.3} There are even number of vertices of odd degree \end{propos}

\begin{defn}[Indegree, Outdegree] G=(V,E) be directed graph and v a vertex of G.\\ 
Indegree of v, $deg^-(v)$ is number of directed edges that end at v.\\
Outdegree of v, $deg^+(v)$ is number of directed edges that originate from v.\\
$\sum_{v\in V}deg^-(v) = \sum_{v\in V}^+(v) = |E|$
\end{defn}
\begin{defn}[Walks] G be graph and v, w be vertices of G.\\
  \textbf{Walk from v to w} is an finite alternating sequence of vertices and edges of G. Walk has the form $v_0e_1v_1e_2...v_{n-1}e_nv_n$, where $v_0=v, v_n=w$. Number of edges n is length of walk (repeat edge/vertex)\\
  \textbf{Trivial Walk from v to v} - Single Vertex v\\
  \textbf{Trail from v to w} - walk without repeated edge\\
  \textbf{Path from v to w} - trail without repeated vertex and edges\\
  \textbf{Closed Walk} - Walk that starts and ends at same vertex (Repeated Vertex)\\
  \textbf{Circuit} - Closed Walk length at least 3 without repeated edge (Repeated Vertex)\\
  \textbf{Simple Circuit} - No repeated vertex except first and last\\
  \textbf{Cyclic} - Loops or cycle, otherwise \textbf{Acyclic}
\end{defn}

\begin{defn}[Connecteddness] Vertices are connected iff walk from v to w. G is connected iff $\forall$ vertices $v, w \in V, \exists$ a walk from v to w. (All vertices are connected)
\end{defn}

\begin{lemma}{10.2.1} Let G be a graph
  \begin{enumerate}
    \item If G is connected, any 2 distinct vertices are connected by path
    \item If v and w are part of circuit in G, and one edge is removed, there exists trail from v to w in G
    \item G is connected and G contains circuit, edge of circuit can be removed without disconnecting G
  \end{enumerate}
\end{lemma}

\begin{defn}[Connected Component] (Subgraph of largest possible size)
  H is connected component iff
  \begin{enumerate}
    \item H is subgraph of G
    \item H is connected
    \item No connected subgraph of G has H as subgraph and contains vertices of edges not in H.
  \end{enumerate}
\end{defn}

\begin{defn}[Euler Circuit] Contains every vertex and traverses every edge exactly once (Can repeat vertices) \end{defn}
\begin{defn}[Euler Graph] Contains Euler Circuit \end{defn}

\begin{theorem}{10.2.2} If graph has euler circuit, ever vertex of graph has positive even degree \end{theorem}
\begin{theorem}{10.2.2} (Contrapositive) If vertex has odd degree, then graph does not have Euler circuit \end{theorem}
\begin{theorem}{10.2.3} G is connected and degree of every vertex of G is even integer, then G has Euler circuit \end{theorem}
\begin{theorem}{10.2.4} G has euler circuit iff G is connected and every vertex has even degree \end{theorem}

\begin{defn}[Euler Trail] passes through every vertex at least one and edge only once \end{defn}

\begin{corollary}{10.2.5} Euler trail from v to w iff G is connected, v and w have odd degree and all other vertices have even degree \end{corollary}

\begin{defn}[Hamiltonian Circuit] Simple circuit that includes every vertex of G (Every vertex appears once\end{defn}
\begin{defn}[Hamilton Graph] Contains Hamilton Circuit \end{defn}

\begin{propos}{10.2.6} If G has Hamiltonian Circuit, G has subgraph H with the following
  \begin{enumerate}
    \item H contains every vertex of G
    \item H is connected
    \item H has same number of edges as vertices
    \item Every vertex of H has degree 2
  \end{enumerate}
\end{propos}

\begin{defn}[Adjacency Matrix] \textbf{A}$ = (a_{ij})$ over the set of non-negative integers s.t. $a_{ij} = $ number of arrows from $v_i$ to $v_j$ \end{defn}

\begin{theorem}{10.3.2}[Number of walks of length n] A is adjacency matrix of G, the ij-th entry of $A^n = $ number of walks of length n from $v_i$ to $v_j$ \end{theorem}

\begin{defn}[Isomorphic Graph] $G=(V_G,E_G)$ and $G' = (V_{G'}, E_{G'})$

  G is isomorphic to $G'$, denoted $G \cong G'$, iff bijections $g: V_G \to V_{G'}$ and $h: E_G \to E_{G'}$, that preserve edge-edgepoint functions of G and G', in sense that $\forall v \in V_G, e \in E_G, v$ is an endpoint of $e \iff g(v)$ is and endpoint of $h(e)$
\end{defn}

\begin{theorem}{10.4.1}[Graph Isomorphism is Equivalence Relation]
  S be set of graphs and let $\cong$ be relation of graph isomorphism on S. $\cong$ is equivalence relation on S
\end{theorem}

\begin{defn}[Planar Graph] is graph that can be drawn on 2D plane without edges crossing
\end{defn}

\begin{theorem}{Kuratowski's Theorem} Planar iff does not contain subgraph that is a subdivision of $K_5$ or complete bipartite $K_{3,3}$ \end{theorem}

\begin{theorem}{Euler's Formula} For planar simple graph, let f be number of faces, $f = |E| - |V| + 2$ \end{theorem}

\subsection*{Trees}

\begin{defn}[Tree] \textbf{Tree} iff circuit free and connected\\
  \textbf{Trivial Tree} iff Single Vertex\\
\textbf{Forest} iff circuit-free and not connected \end{defn}

\begin{lemma}{10.5.1} Non trivial tree has at least one vertex of degree 1 \end{lemma}

\begin{defn}[Terminal Vertex and Internal Vertex] Vertex of degree 1 in T is terminal vertex, vertex of degree greater than 1 is internal vertex \end{defn}

\begin{theorem}{10.5.2} Any tree with n vertices $(n > 0)$ has $n-1$ edges \end{theorem}

\begin{defn}E.g. Find all non-isomorphic trees with 4 vertices
  4 vertices means 3 edges = total degree of 6. So $deg(a) + deg(b) + deg(c) + deg(d) = 6$
\end{defn}

\begin{lemma}{10.5.3} G is connected graph, C is any circuit, one of the edges of C is removed from G, the graph remains connected \end{lemma}

\begin{theorem}{10.5.4} G is a connected graph with n vertices and n-1 edges, G is a tree \end{theorem}

\begin{defn}[Rooted Tree, Level, Height] \textbf{Rooted tree} is a tree with 1 vertex distinguished from others called root\\
  \textbf{Level} of a vertex is no of edges between it and root\\
  \textbf{Height} of a rooted tree is max level of any vertex of the tree
\end{defn}

\begin{defn}[Child, Parent, Sibling, Ancestor, Descendant]
  \textbf{Children} of v are all vertices that are adjacent to v and 1 level farther away from the root than v\\
  \textbf{Parent} if w is a child of v, then v is parent of w, and 2 vertices that are both children of same parent is \textbf{siblings} \\
  \textbf{Ancestor} if v lies on unique path between w and root, v is ancestor of w, and w is \textbf{descendant} of v\\
\end{defn}

\begin{defn}[Binary Tree, Full Binary Tree]
  \textbf{Binary Tree} is rootred tree with every parent at most 2 children. Each child is either left child or right child.\\
  \textbf{Full Binary Tree} is where every parent has exactly 2 children
\end{defn}

\begin{defn}[Left Subtree] Root is the left tree of v, vertices consist of left child o v and all its descendants, whose edges consist of all those edges of T that connect vertices of left subtree
\end{defn}

\begin{theorem}{10.6.1}[Full Binary Tree Theorem] If T is full binary tree with k internal vertices, then T has total of $2k+1$ vertices, and has $k+1$ terminal vertices (leaves \end{theorem}

\begin{theorem}{10.6.2} non-negative integers h, if T is any binary tree with height h and terminal vertices (leaves), then $t \leq 2^h$, $\log_2t \leq h$
\end{theorem}

\begin{defn}[Breadth-First Search] Starts at root, visit adjacent vertices, and then next level \end{defn}

\begin{defn}Depth-First Search \\
  \textbf{Pre-order} Print root, traverse left, traverse right \\
  \textbf{In-order} Traverse Left, Print Root, Traverse right\\
  \textbf{post-order} Traverse Left, Traverse Right, Print Root
\end{defn}

\begin{defn}[Spanning Tree] Subgraph that contains every vertex of G and is a tree \end{defn}

\begin{propos}{10.7.1}
  \begin{enumerate}
    \item Every connected graph has a spanning tree
    \item Any 2 spanning trees for a graph have same number of edges
  \end{enumerate}
\end{propos}

\begin{defn} Weighted Graph and Minimum Spanning Tree\\
  \textbf{Weighted Graph} is a graph for which each edge has a positive real number weight. Total weight = sum of weights of all edges\\
  \textbf{Minimum Spanning Tree} Least possible total weight compared to all other spanning trees for graph
\end{defn}

\begin{theorem}{Kruskal's Algorithm}, Input is a connected weighted graph with n vertices
  \begin{enumerate}
    \item Initialise T to have all vertices of G and no edges
    \item Let E be set of Edges in G and m = 0

    \item While $(m < n - 1)$
      \begin{enumerate}
        \item Find e in E of least weight
        \item Delete e from E
        \item If adding e to T does not create circuit, add e to T and set m = m + 1
      \end{enumerate}
  \end{enumerate}
\end{theorem}

\begin{theorem}{Prim's Algorithm}Input is a connected weighted graph with n vertices
  \begin{enumerate}
    \item Pick vertex v of G and let T be graph with this vertex only
    \item Let V be set of all vertices of G except v
    \item For i = 1 to n - 1
      \begin{enumerate}
        \item Find edge $e$ of G s.t. e connects T to one vertice in V, e has the least weight of all edges connecteing T to V. Let w be endpoint of e in V
        \item Add e and w to T, delete w from V
      \end{enumerate}
  \end{enumerate}
\end{theorem}

\end{document}
